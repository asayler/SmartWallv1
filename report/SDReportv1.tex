%Andy Sayler
%EE126
%Lab1

\documentclass[12pt]{article}
\usepackage[text={6.5in, 9in}, centering]{geometry}
\usepackage{graphicx}
\usepackage{listings}
\lstloadlanguages{C}
\usepackage{amsmath}
\usepackage{url}

%\usepackage{tikz}
%\usepackage{verbatim}

\lstset{
  language=C,
  basicstyle=\footnotesize,
  numbers=left,
  numberstyle=\footnotesize,
  stepnumber=1,
  numbersep=5pt,
  showspaces=false,
  showstringspaces=false,
  showtabs=false,
  tabsize=4,
  captionpos=b,
  breaklines=false,
  breakatwhitespace=false,
  title=\lstname,
  frame=single,
  frameround=tttt,
}

\title{SmartWall: SmartOutlet}
\author{Andy Sayler, Laura Costello, Patrick McKelvy}
\date{\today}

\begin{document}

\maketitle
    
\begin{abstract}

Our team set out to design a system to enable the energy monitoring and
automated control of a collection of in-wall electrical devices and plug-in
appliances. To achieve this goal, we designed the SmartWall protocol,
a self-configuring application layer network protocol for querying and
controlling electrical devices. To demonstrate the power of this protocol
to enable communication and control of in-wall electrical devices, we
designed and prototyped a SmartOutlet. The SmartOutlet is an
intelligent drop-in
replacement for the standard electrical wall outlet that adds the
ability to remotely toggle the outlet plugs on and off and to monitor the
energy usage of any device utilizing the outlet. Finally, we designed a
web based user interface for controlling, administering, and
monitoring our SmartWall network (and any SmartOutlets on it). We hope
this project demonstrates the benefits that come from embedding
standardized energy monitoring and remote control capabilities in
endpoint electrical grid hardware and devices and giving the end user
thaw ability to control and monitor these devices.
  
\end{abstract}

\tableofcontents

\pagebreak

\section{Introduction}



\section{Background}


\section{Design}


\section{Solution}


\section{Results}


\section{Analysis}


\section{Conclusion}


\section{Recommendations}


\renewcommand{\refname}{\section{References}}
\begin{thebibliography}{10}

\bibitem{RFC1958} Internet Engineering Task Force: Network Working Group.
  \newblock ``RFC 1958''.
  \newblock Brian Carpenter, Editor.
  \newblock (June 1996).
  \newblock \url{http://datatracker.ietf.org/doc/rfc1958/}.

\bibitem{CNID} Metcalfe, RM.
  \newblock ``Computer Network Interface Design: Lessons from
  Arpanet and Ethernet''.
  \newblock IEEE Journal on Selected Areas in Communications.
  \newblock (Feburary 1993).

\bibitem{artUnix} Raymond, Eric S.
  \newblock \emph{The Art of Unix Programming}.
  \newblock Boston: Addison-Wesley, 2004.

\bibitem{compNetworks} Tanenbaum, Andrew S.
  \newblock \emph{Computer Networks}. 4th ed.
  \newblock Upper Saddle River, NJ: Prentice Hall PTR, 2003.

\bibitem{googleStory} Vise, David A., and Mark Malseed.
  \newblock \emph{The Google Story}.
  \newblock New York, NY:Delacorte, 2008.
  \newblock Google Books.
  \newblock \url{http://books.google.com}.

\bibitem{UIDesign} Wood, Larry E.
  \newblock \emph{User Interface Design: Bridging the Gap from User
    Requirements to Design}.
  \newblock Boca Raton: CRC, 1998.
  \newblock Google Books.
  \newblock \url{http://books.google.com>}.

\end{thebibliography}


\pagebreak

\section{Appendix A - }

\pagebreak

\end{document}
