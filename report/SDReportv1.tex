%Andy Sayler
%EE126
%Lab1

\documentclass[12pt]{article}
\usepackage[text={6.5in, 9in}, centering]{geometry}
\usepackage{graphicx}
\usepackage{listings}
\lstloadlanguages{C}
\usepackage{amsmath}
\usepackage{url}

%\usepackage{tikz}
%\usepackage{verbatim}

\lstset{
  language=C,
  basicstyle=\footnotesize,
  numbers=left,
  numberstyle=\footnotesize,
  stepnumber=1,
  numbersep=5pt,
  showspaces=false,
  showstringspaces=false,
  showtabs=false,
  tabsize=4,
  captionpos=b,
  breaklines=false,
  breakatwhitespace=false,
  title=\lstname,
  frame=single,
  frameround=tttt,
}

\title{SmartWall: SmartOutlet}
\author{Andy Sayler, Laura Costello, Patrick McKelvy}
\date{\today}

\begin{document}

\maketitle
    
\begin{abstract}

Our team set out to design a system to enable the energy monitoring and
automated control of a collection of in-wall electrical devices and plug-in
appliances. To achieve this goal, we designed the SmartWall protocol,
a self-configuring application layer network protocol for querying and
controlling electrical devices. To demonstrate the power of this protocol
to enable communication and control of in-wall electrical devices, we
designed and prototyped a SmartOutlet. The SmartOutlet is an
intelligent drop-in
replacement for the standard electrical wall outlet that adds the
ability to remotely toggle the outlet plugs on and off and to monitor the
energy usage of any device utilizing the outlet. Finally, we designed a
web based user interface for controlling, administering, and
monitoring our SmartWall network (and any SmartOutlets on it). We hope
this project demonstrates the benefits that come from embedding
standardized energy monitoring and remote control capabilities in
endpoint electrical grid hardware and devices and giving the end user
the ability to control and monitor these devices.
  
\end{abstract}

\pagebreak

\tableofcontents

\pagebreak

\section{Introduction}

Currently, residential and business electricity users lack a simple
and affordable means to monitor their energy usage on a per device
level and automate or remotely control electrical devices and
hardware. Even in the 21st century world where most facets of our
lives are accessible form the web and where we can instantaneously
communicate across the globe, we have failed to extend this
connectivity and remote functionality to the basic appliances, devices,
and electrical hardware we use every day. We can process stock trading
orders from our cell phones, or track airline flights from our
laptops, but we can not use these same devices or technologies to turn
the lights out while we're at work or tell us how much energy a
particular appliance is consuming.

We aim to remedy this technological deficit
by developing a system capable of providing
power monitoring, automation, and remote control
to the end user of any device, appliance, or electrical fixture that
connects to the local electrical system. We call this system
``SmartWall'', reflecting a combination of the current trend
toward assigning an intelligence qualifier to modern ubiquitous
interconnected technologies (\emph{smart}) and the fact that
current electrical systems
revolve around the in-\emph{wall} electrical wiring present in most homes and
office buildings.

The goals of a SmartWall system are as follow:
\begin{description}
  \setlength{\itemsep}{0pt}
  \setlength{\parskip}{0pt}
  \setlength{\parsep}{0pt}
\item[Convenience:] Providing end users with the ability automate and
  control electrically connected devices in their homes from their
  computer, cell phone, or similar device is a desirable convenience
  and one that is currently lacking. From automatically turning lights
  on and off to primitively preheating the oven or turning up the heat
  form work before one arrives home, the SmartWall system and add
  considerable convince and functionality to one's day-today
  interaction with the devices and appliance they use.
\item[Energy Conservation:] By providing users with the ability to
  monitor their energy usage on a fine grain level and providing them
  with the ability to programmaticly control the behavior of their electrical
  devices, users have all the necessary tools required for analysis
  and reducing energy waste. This includes everything from
  automatically turning off unnecessary devices when one is at work
  during the day to identifying the largest sources of energy use and
  addressing these sources individually.
\item[Integration Framework:] The SmartWall system lays the groundwork for an
  endless number of integration opportunities with other
  technologies. By providing programmatic and network accessible
  control of the electrical devices and appliances in one's home or
  office, we open the door for providing a considerable service to any
  number of external systems. From integration with Smart-Grid meters
  that require the ability to turn large appliances on or off at the
  power companies discretion to providing landlords and building
  managers the ability to track energy usage on a per person level and
  bill their tenets accordingly, the possibilities for external
  integration are immense.
\end{description}

The SmartWall vision has three facets:
\begin{description}
  \setlength{\itemsep}{0pt}
  \setlength{\parskip}{0pt}
  \setlength{\parsep}{0pt}
\item[SmartWall Protocol:] The underlying network protocol that
  facilitates communication with and control of SmartWall devices.
\item[SmartWall Enabled Devices:] Any device or electrical fixture that
  connects to the local electrical system and is capable of
  communicating via the SmartWall protocol.
\item[SmartWall User Interfaces:] Any device or system that connects
  the end user to their local SmartWall network for the purpose of
  monitoring, controlling, or automating devices on the network.
\end{description}

In line with these three facets, we will be  developing our solution
from three distinct directions:
\begin{description}
  \setlength{\itemsep}{0pt}
  \setlength{\parskip}{0pt}
  \setlength{\parsep}{0pt}
\item[Network:] The SmartWall Network Protocol and Standard
\item[Hardware:] A Proof-of-Concept SmartWall Device - The SmartOutlet
\item[User Interface:] A Proof-of-Concept Web Based User Interface 
\end{description}

Network protocol development involves designing and implementing the
SmartWall protocol governing the communications and command set supported
by our SmartOutlet and our wider SmartWall system suite
vision. We aim for the SmartWall protocol to be open and
extensible. It will fit into the standard IP network stack as an application
layer protocol utilizing the UDP transport layer protocol on top of a
wire, wireless, or power-line Ethernet implementation.

Hardware prototype development encompasses designing,
building, and testing an actual SmartOutlet prototype. The SmartOutlet
device will be a drop-in replacement for standard in-wall electrical
devices with the added benefits of providing the ability to switch
each socket on or off and
to monitor the power and energy usage of each socket via the SmartWall
network.

User interface development encompasses creating a web based user
interface, as well as any intermediate software that will be necessary
to allow our interface to communicate with our prototype
SmartOutlet(s) via the SmartWall network. This interface will provide
the end user with a simple way to monitor and control their power
consumption on a per device level. It will also allow users the
convenience of automating and remotely controlling devices in their home.

These three sections encompass the necessary tasks that were completed
in reaching our project goal and also reflect the division of labor
between the members of our team. We will reference these three
approaches in the sections below.

\section{Background}

\subsection{Existing Technology}

While the SmartWall system aims to provide energy monitoring and
control capabilities to a wide range of electrical devices, it is not the
first system to offer these services. Several preexisting technologies
of various levels of success already exist in this arena.

The promise of home automation has existed since digital communications
begin to gain prominence in the 1970s. The most extensive effort in this
field is the X10 power-line based automated control system
\cite{wiki-X10}. The X10 system, however, has a number of detractors
that have prevented it form becoming universally deployed:
\begin{itemize}
  \setlength{\itemsep}{0pt}
  \setlength{\parskip}{0pt}
  \setlength{\parsep}{0pt}
\item It is not easily extensible, making ti very difficult to add
  support for new devices or for functions beyond simple on/off, dim,
  etc controls.
\item It uses outdated and error-prone power-line communication
  standards making it unsuitable for sending commands over any network
  other than the local power system and causing frequent errors.
\item It does not support power monitoring or more advanced state querying.
\end{itemize}  

There are also a number of products on the market that allow the user
to monitor the energy usage of their devices. P3 Internationale's
``Kill A Watt'' product is a good example of one such a device
\cite{p3-KillAWatt}. Like the X10 system, however, such devices have
failed to gain prominence due to a number of factors including:
\begin{itemize}
  \setlength{\itemsep}{0pt}
  \setlength{\parskip}{0pt}
  \setlength{\parsep}{0pt}
\item They are external, 3rd party devices requiring the consumer to
  purchase an additional power monitor for each electrical device they
  wish to monitor. 
\item They generally provide no networking or remote monitoring capabilities.
\item They are passive monitors, proving no control over devices.
\end{itemize}  

Our SmartWall system combines the benefits of existing automation and
monitoring technologies while addressing their shortcomings and
approaching both capabilities form a 21st century, network-centric approach.

\subsection{Networking}
Ethernet based IP networks have cemented themselves as the standard
for modern communication systems. The IP network stack provides an
abstract, layered approach to network communication. Each layer of the
stack implements specific functionality by relying on the layers below
it and by providing a service to the layers above it. Thus, we can
separate the physical layer, link layer, Internet layer, transport
layer, Adan application layer and focus on each independently.

The IEEE 1901 ``Homeplug'' collection of power-line Ethernet protocols are
quickly becoming the standard for power-line communications and provide
us with an abstracted power-line communication physical layer on which our
SmartWall system can ride.

The UDP transportation layer provides lightweight uni-cast, multi-cast, and
broadcast datagram communication ideal. It has a simple and easy to
implement interface and, along with TCP, is one of the two most common
transportation layers available in the IP stack.

\subsection{Hardware}

\subsection{User Interface}
No matter how excellent the underlying technology, if the potential
users do not find a product’s interface to be usable, that product is
likely to fail. Thus, a key element of a successful product is a
successful user interface. For something of a case study, we examined the
wild success of Google. Much of Google's appeal is based upon the fact
that “its clean interface made people feel good about the product,
one that always seemed to work properly” \cite{googleStory}. These are
two of the major concerns in any user project: a product that works
properly, and an
interface that the users can feel good about. Thus, interfaces are
all about usability. To be usable, an interface must provide its
potential users with “access to the functions and features of an
application in a way that reflects the users’ ways of thinking about
the tasks” and “means...to interact with the application in ways that
are intuitive and natural” \cite{UIDesign}.

\section{Design}

Thus, our SmartWall protocol will
leverage the standard UDP/IP network stack.
This has the benefit of
leveraging a wide range of well understood existing hardware and
software. It also means that we can leverage existing physical layer
technologies to obtain power-line based Ethernet connectivity.

\section{Solution}



\section{Results}


\section{Analysis}


\section{Conclusion}


\section{Recommendations}


\renewcommand{\refname}{\section{References}}
\begin{thebibliography}{10}

\bibitem{RFC1958} Internet Engineering Task Force: Network Working Group.
  \newblock ``RFC 1958''.
  \newblock Brian Carpenter, Editor.
  \newblock (June 1996).
  \newblock \url{http://datatracker.ietf.org/doc/rfc1958/}.

\bibitem{CNID} Metcalfe, RM.
  \newblock ``Computer Network Interface Design: Lessons from
  Arpanet and Ethernet''.
  \newblock IEEE Journal on Selected Areas in Communications.
  \newblock (Feburary 1993).

\bibitem{artUnix} Raymond, Eric S.
  \newblock \emph{The Art of Unix Programming}.
  \newblock Boston: Addison-Wesley, 2004.

\bibitem{p3-KillAWatt} P3 International.
  \newblock ``P4400 Kill A Watt''.
  \newblock Retrieved 2004 EST, May 1, 2011.
  \newblock \url{http://www.p3international.com/products/special/P4400/P4400-CE.html}

\bibitem{compNetworks} Tanenbaum, Andrew S.
  \newblock \emph{Computer Networks}. 4th ed.
  \newblock Upper Saddle River, NJ: Prentice Hall PTR, 2003.

\bibitem{googleStory} Vise, David A., and Mark Malseed.
  \newblock \emph{The Google Story}.
  \newblock New York, NY:Delacorte, 2008.
  \newblock Google Books.
  \newblock \url{http://books.google.com}.

\bibitem{wiki-X10} ``X10 (industry standard)''.
  \newblock Wikipedia, The Free Encyclopedia.
  \newblock 2011, April 29.  Retrieved 1953 EST, May 1, 2011.
  \newblock \url{http://en.wikipedia.org/w/index.php?title=X10_(industry_standard)&oldid=426496014}.

\bibitem{UIDesign} Wood, Larry E.
  \newblock \emph{User Interface Design: Bridging the Gap from User
    Requirements to Design}.
  \newblock Boca Raton: CRC, 1998.
  \newblock Google Books.
  \newblock \url{http://books.google.com>}.

\end{thebibliography}


\pagebreak

\section{Appendix A - }

\pagebreak

\end{document}
